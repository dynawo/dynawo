%% Copyright (c) 2015-2020, RTE (http://www.rte-france.com)
%% See AUTHORS.txt
%% All rights reserved.
%% This Source Code Form is subject to the terms of the Mozilla Public
%% License, v. 2.0. If a copy of the MPL was not distributed with this
%% file, you can obtain one at http://mozilla.org/MPL/2.0/.
%% SPDX-License-Identifier: MPL-2.0
%%
%% This file is part of Dynawo, an hybrid C++/Modelica open source suite of simulation tools for power systems.

\documentclass[a4paper, 12pt]{report}

%% Except where otherwise noted, content in this documentation is Copyright (c)
%% 2015-2019, RTE (http://www.rte-france.com) and licensed under a
%% CC-BY-4.0 (https://creativecommons.org/licenses/by/4.0/)
%% license. All rights reserved.

% Latin Modern fam­ily of fonts
\usepackage{lmodern}

\usepackage[english]{babel}

% specify encoding
\usepackage[utf8]{inputenc} % input
\usepackage[T1]{fontenc} % output

% Document structure setup
\usepackage{titlesec} % To change chapter format
\setcounter{tocdepth}{3} % Add subsubsection in Content
\setcounter{secnumdepth}{3} % Add numbering for subsubsection
\setlength{\parindent}{0pt} % No paragraph indentation

% Change title format for chapter
\titleformat{\chapter}{\Huge\bf}{\thechapter}{20pt}{\Huge\bf}

% To add links on page number in Content and hide red rectangle on links
\usepackage[hidelinks, linktoc=all]{hyperref}
\usepackage[nottoc]{tocbibind}  % To add biblio in table of content
\usepackage{textcomp} % For single quote
\usepackage{url} % Allow linebreaks in \url command
\usepackage{listings} % To add code samples

% Default listings parameters
\lstset
{
  aboveskip={1\baselineskip}, % a bit of space above
  backgroundcolor=\color{shadecolor}, % choose the background color
  basicstyle={\ttfamily\footnotesize}, % use font and smaller size \small \footnotesize
  breakatwhitespace=true, % sets if automatic breaks should only happen at whitespace
  breaklines=true, % sets automatic line breaking
  columns=fixed, % nice spacing -> fixed / flexible
  mathescape=false, % escape to latex false
  numbers=left, % where to put the line-numbers
  numberstyle=\tiny\color{gray}, % the style that is used for the line-numbers
  showstringspaces=false, % do not emphasize spaces in strings
  tabsize=4, % number of spaces of a TAB
  texcl=false, % activates or deactivates LaTeX comment lines
  upquote=true % upright quotes
}

% Avoid numbering starting at each chapter for figures
\usepackage{chngcntr}
\counterwithout{figure}{chapter}

\usepackage{tikz} % macro pack­age for cre­at­ing graph­ics
\usepackage{pgfplots} % draws func­tion plots (based on pgf/tikz)

\usepackage{algorithm} % Add algorithms
\usepackage[noend]{algpseudocode} %  all end ... lines are omitted in algos

\usepackage{amsmath} % Add math­e­mat­i­cal fea­tures
\usepackage{schemabloc} % Add block diagram library (french one)

\usepackage{adjustbox} % Add box for flowchart

\usepackage{booktabs} % for toprule and midrule in tables

\usepackage{tabularx}

\usepackage[nolist]{acronym} % don’t write the list of acronyms.
% Acronyms list
\begin{acronym}
\acro{BDF}{Backward Differentiation Formula}
\acro{BE}{Backward Euler}
\acro{DAE}{Differential Algebraic Equations}
\acro{IDA}{Implicit Differential-Algebraic solver}
\acro{LLNL}{Lawrence Livermore National Lab}
\acro{KINSOL}{Krylov Inexact Newton SOLver}
\acro{NR}{Newton-Raphson}
\acro{PLL}{Phase-Locked Loop}
\acro{SVC}{Static Var Compensator}
\acro{SUNDIALS}{SUite of Nonlinear and DIfferential/ALgebraic equation Solvers}
\acro{WECC}{Western Electricity Coordinating Council}
\end{acronym}

% Syntax highlight
%% Except where otherwise noted, content in this documentation is Copyright (c)
%% 2015-2019, RTE (http://www.rte-france.com) and licensed under a
%% CC-BY-4.0 (https://creativecommons.org/licenses/by/4.0/)
%% license. All rights reserved.

\usepackage{color}

\definecolor{blue}{rgb}{0,0,1}
\definecolor{lightblue}{rgb}{.3,.5,1}
\definecolor{darkblue}{rgb}{0,0,.4}
\definecolor{red}{rgb}{1,0,0}
\definecolor{darkred}{rgb}{.56,0,0}
\definecolor{pink}{rgb}{.933,0,.933}
\definecolor{purple}{rgb}{0.58,0,0.82}
\definecolor{green}{rgb}{0.133,0.545,0.133}
\definecolor{darkgreen}{rgb}{0,.4,0}
\definecolor{gray}{rgb}{.3,.3,.3}
\definecolor{darkgray}{rgb}{.2,.2,.2}
\definecolor{shadecolor}{gray}{0.925}

% **********************************************************************************
% Syntax : Bash (bash)
% **********************************************************************************

\lstdefinelanguage{bash}
{
  keywordstyle=\color{blue},
  morekeywords={
    cd,
    export,
    source},
  numbers=none,
  deletekeywords={jobs}
}

% **********************************************************************************
% Syntax : XML
% **********************************************************************************

\lstdefinelanguage{XML}
{
  morestring=[s][\color{purple}]{"}{"},
  morecomment=[s][\color{green}]{<?}{?>},
  morecomment=[s][\color{green}]{<!--}{-->},
  stringstyle=\color{black},
  identifierstyle=\color{blue},
  keywordstyle=\color{red},
  morekeywords={
    xmlns,
    xsi,
    noNamespaceSchemaLocation,
    type,
    source,
    target,
    version,
    tool,
    transRef,
    roleRef,
    objective,
    eventually}
}

% **********************************************************************************
% Syntax : Modelica (modelica)
% **********************************************************************************
\lstdefinelanguage{Modelica}{
  alsoletter={...},
  morekeywords=[1]{ % types
      Boolean,
      Integer,
      Real},
  keywordstyle=[1]\color{red},
  morekeywords=[2]{ % keywords
    algorithm,
    and,
    annotation,
    assert,
    block,
    class,
    connector,
    constant,
    discrete,
    else,
    elseif,
    elsewhen,
    end,
    equation,
    exit,
    extends,
    external,
    false,
    final,
    flow,
    for,
    function,
    if,
    in,
    inner,
    input,
    import,
    loop,
    model,
    nondiscrete,
    not,
    or,
    outer,
    output,
    package,
    parameter,
    public,
    protected,
    record,
    redeclare,
    replaceable,
    return,
    size,
    terminate,
    then,
    true,
    type,
    when,
    while},
  keywordstyle=[2]\color{darkred},
  morekeywords=[3]{ % functions
    abs,
    acos,
    asin,
    atan,
    atan2,
    Complex,
    connect,
    conj,
    cos,
    cosh,
    cross,
    der,
    edge,
    exp,
    fromPolar,
    imag,
    noEvent,
    pre,
    sign,
    sin,
    sinh,
    sqrt,
    tan,
    tanh},
  keywordstyle=[3]\color{blue},
  morecomment=[l][\color{green}]{//}, % comments
  morecomment=[s][\color{green}]{/*}{*/}, % comments
  morestring=[b][\color{pink}]{'}, % strings
  morestring=[b][\color{pink}]{"}, % strings
}


\usepackage{xspace} % Define typography
\usepackage{dirtree}
\newcommand{\Dynawo}[0]{Dyna$\omega$o\xspace}


\begin{document}

\chapter{Static Var Compensator - Infinite Bus}

The \{Static Var Compensator - Infinite Bus\} test case is a simplified system used to illustrate the behaviour of the Static Var Compensator model present in the \Dynawo library on events such as a voltage reference change or load variation.

\section{Test case description}

The following test case consists of a static var compensator connected to an infinite bus through a line as presented in Figure ~\ref{TestCase}.

\begin{figure}[H]
\centering
\def\factor{0.4}
\begin{tikzpicture}[every node/.style={inner sep=0,outer sep=0}]
% Infinite bus
\path (0,0)  pic[scale=0.4,local bounding box=bus] {infinite bus};
% Generator
\path (8,0) pic[scale=0.2,local bounding box=svarc] {SVarC};
% Line 1
\draw (bus.east) -- (svarc.west);
% Bus inf
\draw (bus.east) ++ (0,0.6) --++ (0,-1.2);
% Bus tfo 2
\draw (svarc.west) ++ (-1,0.6) --++ (0,-1.2);
\end{tikzpicture}
\caption{Static Var Compensator - Infinite Bus system representation}
\label{TestCase}
\end{figure}

\subsection{Initial Conditions}

The static var compensator base voltage is $UNom = 225 kV$. \\

The static var compensator apparent power is $SNom = 100 MVA$.\\

The line parameters in per unit on 100 MVA base are:
\begin{center}
\begin{tabular}{l|l}
   $R = 0 p.u$ & $X = 0.027654 p.u$  \\
\end{tabular}
\end{center}

At $t=0s$, the static var compensator injects 0 MW and 0 Mvar, so its terminal voltage is 1 p.u.

\subsection{Model}

The static var compensator is modeled using a simplified SVarCPV model. Its control has no dynamic.

The static var compensator can be seen as a variable admittance and is thus modelled thanks to the susceptance B.\\

Its injected complex current i is computed thanks to the susceptance B and the complex voltage at its terminal u with the following equation:
\[
\begin{aligned}
& i = j*B*u \\
\end{aligned}
\]

The susceptance B is given by the model.

The susceptance represents the reactive power contribution of the static var compensator. The susceptance computation in per-unit depends on the static var compensator regulation mode:

\begin{itemize}
\item if mode = OFF, $B_{Pu} = 0$. The static var compensator doesn't inject reactive power;
\item if mode = STANDBY, $B_{Pu} = BShunt_{Pu}$.
The static var compensator behaves as a  capacitor;
\item if mode = RUNNING\_V, $B_{Pu} = BVar_{Pu} + BShunt_{Pu}$. The susceptance in per-unit $B_{Pu}$ is composed of a variable susceptance $BVar_{Pu}$ and a fixed susceptance $BShunt_{Pu}$. The fixed susceptance $BShunt_{Pu}$ represents the susceptance in standby mode, and the variable susceptance $BVar_{Pu}$ represents the equivalent variable susceptance generated by the static var compensator power electronics and enabling to control the voltage.
\end{itemize}

The susceptance regulation implements the following regulation law:
\[
\begin{aligned}
& URef_{Pu} = U_{Pu} + Lambda*B_{Pu}*U_{Pu}^2\\
\end{aligned}
\]

The variable susceptance $BVar_{Pu}$ is limited thanks to limitations taking into account static susceptance limits.

\subsection{Scenarios}
The simulated scenarios are :
\begin{itemize}
\item a step on the reference voltage;
\item a small reactive load variation;
\end{itemize}

\newpage
\section{Results}

\subsection{Step on the reference voltage}

The SVC is initially running. A step on URef from 225 kV to 230kV is realised at $t=10s$.

\begin{figure}[H]
\subfigure[Voltage (kV)]
{%
  \begin{tikzpicture}
    \begin{axis}[height = 1.7in]
       \addplot[color=blue!50]
       table[x=time,y expr=\thisrow{Test_svarc_UPu}*225]
       {../SVarC_1_StepUref/reference/outputs/curves/curves.csv};
    \end{axis}
  \end{tikzpicture}
}
\subfigure[Reactive power (MVar)]
{%
  \begin{tikzpicture}
    \begin{axis}[height = 1.7in]
       \addplot[color=blue!50]
       table[x=time,y expr=\thisrow{Test_svarc_QInjPu}*100]
       {../SVarC_1_StepUref/reference/outputs/curves/curves.csv};
    \end{axis}
  \end{tikzpicture}
}
\subfigure[Susceptance (p.u)]
{%
  \begin{tikzpicture}
    \begin{axis}[height = 1.7in]
        \addplot[color=blue!50]
       table[x=time,y expr=\thisrow{Test_svarc_BPu}]
       {../SVarC_1_StepUref/reference/outputs/curves/curves.csv};
    \end{axis}
  \end{tikzpicture}
}
\caption{Step on the reference voltage}
\end{figure}

At $t=10s$, the SVC starts providing reactive power to follow the voltage reference change.

\newpage
\subsection{Reactive load variation}

The SVC is initially in the standby mode. A reactive load variation of Q=150 Mvar is realised at $t=10s$.

\begin{figure}[H]
\subfigure[Voltage (kV)]
{%
  \begin{tikzpicture}
    \begin{axis}[height = 1.7in]
        \addplot[color=blue!50]
       table[x=time,y expr=\thisrow{SVarC_SVarC_UPu}*225]
       {../SVarC_2_LoadVarQ/reference/outputs/curves/curves.csv};
    \end{axis}
  \end{tikzpicture}
}
\subfigure[Reactive power (MVar)]
{%
  \begin{tikzpicture}
    \begin{axis}[height = 1.7in]
        \addplot[color=blue!50]
       table[x=time,y expr=\thisrow{SVarC_SVarC_QInjPu}*100]
       {../SVarC_2_LoadVarQ/reference/outputs/curves/curves.csv};
    \end{axis}
  \end{tikzpicture}
}
\subfigure[Susceptance (p.u)]
{%
  \begin{tikzpicture}
    \begin{axis}[height = 1.7in]
        \addplot[color=blue!50]
       table[x=time,y expr=\thisrow{SVarC_SVarC_BPu}]
       {../SVarC_2_LoadVarQ/reference/outputs/curves/curves.csv};
    \end{axis}
  \end{tikzpicture}
}
\caption{Small reactive load variation}
\end{figure}

At $t=10s$, the voltage suddenly drops due to the load variation and crosses UthresholdDown (218 kV). As a consequence, the SVC switches to running mode and starts providing reactive power to support the voltage.

\end{document}
