%% Copyright (c) 2025, RTE (http://www.rte-france.com)
%% See AUTHORS.txt
%% All rights reserved.
%% This Source Code Form is subject to the terms of the Mozilla Public
%% License, v. 2.0. If a copy of the MPL was not distributed with this
%% file, you can obtain one at http://mozilla.org/MPL/2.0/.
%% SPDX-License-Identifier: MPL-2.0
%%
%% This file is part of Dynawo, a hybrid C++/Modelica open source suite
%% of simulation tools for power systems.

\documentclass[a4paper, 12pt]{report}


%%  Copyright (c) 2015-2019, RTE (http://www.rte-france.com)
%%  See AUTHORS.txt
%%  All rights reserved.
%%  This Source Code Form is subject to the terms of the Mozilla Public
%%  License, v. 2.0. If a copy of the MPL was not distributed with this
%%  file, you can obtain one at http://mozilla.org/MPL/2.0/.
%%  SPDX-License-Identifier: MPL-2.0
%%
%%  This file is part of Dynawo, a hybrid C++/Modelica open source time domain
%%  simulation tool for power systems.


%%%%%%%%%%%%%%%%%%%%%%%%%%%%%%%%%%%%%%%%%%%
% Define text and document settings
%%%%%%%%%%%%%%%%%%%%%%%%%%%%%%%%%%%%%%%%%%%

\usepackage{lmodern} % Latin Modern fam­ily of fonts
\usepackage[english]{babel} % English

% Specify encoding
\usepackage[utf8]{inputenc} % Input
\usepackage[T1]{fontenc} % Output

% Document structure setup
\usepackage{titlesec} % To change chapter format
\setcounter{tocdepth}{3} % Add subsubsection in Content
\setcounter{secnumdepth}{3} % Add numbering for subsubsection
\setlength{\parindent}{0pt} % No paragraph indentation

% Avoid numbering starting at each chapter for figures
\usepackage{chngcntr}
\counterwithout{figure}{chapter}

% Change title format for chapter
\titleformat{\chapter}{\Huge\bf}{\thechapter}{20pt}{\Huge\bf}

% To add links on page number in Content and hide red rectangle on links
\usepackage[hidelinks, linktoc=all]{hyperref}
\usepackage[nottoc]{tocbibind} % To add biblio in table of content
\usepackage{textcomp} % For single quote
\usepackage{url} % Allow linebreaks in \url command
\usepackage{listings} % To add code samples

% Define typography
\usepackage{xspace}
\usepackage{dirtree}
\newcommand{\Dynawo}[0]{Dyna$\omega$o\xspace}

% Default listings parameters
\lstset
{
  aboveskip={1\baselineskip}, % A bit of space above
  backgroundcolor=\color{shadecolor}, % Choose the background color
  basicstyle={\ttfamily\footnotesize}, % Use font and smaller size \small \footnotesize
  breakatwhitespace=true, % Sets if automatic breaks should only happen at whitespace
  breaklines=true, % Sets automatic line breaking
  columns=fixed, % Nice spacing -> fixed / flexible
  mathescape=false, % Escape to latex false
  numbers=left, % Where to put the line-numbers
  numberstyle=\tiny\color{gray}, % The style that is used for the line-numbers
  showstringspaces=false, % Do not emphasize spaces in strings
  tabsize=4, % Number of spaces of a TAB
  texcl=false, % Activates or deactivates LaTeX comment lines
  upquote=true % Upright quotes
}

% Package for creating tables
\usepackage{array}

%%%%%%%%%%%%%%%%%%%%%%%%%%%%%%%%%%%%%%%%%%%
% Define plots settings
%%%%%%%%%%%%%%%%%%%%%%%%%%%%%%%%%%%%%%%%%%%

% Macro pack­age for cre­at­ing graph­ics
\usepackage{tikz}
\usepackage{subfigure}
\usepackage{float}

% Draws func­tion plots (based on pgf/tikz)
\usepackage{pgfplots}
\pgfplotsset{enlarge x limits=false, xlabel={\begin{small}$time$ (s)\end{small}}, height=0.6\textwidth, width=1\textwidth,
yticklabel style={text width={width("$-0.6$")},align=right}}
\pgfplotstableset{col sep=semicolon}

% Define colors
\usepackage{color}
\definecolor{blue}{rgb}{.3,.5,1}
\definecolor{deepblue}{rgb}{0,0,1}
\definecolor{darkblue}{rgb}{0,0,.4}
\definecolor{red}{rgb}{1,0,0}
\definecolor{darkred}{rgb}{.56,0,0}
\definecolor{pink}{rgb}{.933,0,.933}
\definecolor{purple}{rgb}{0.58,0,0.82}
\definecolor{green}{rgb}{0.133,0.545,0.133}
\definecolor{darkgreen}{rgb}{0,.4,0}
\definecolor{gray}{rgb}{.3,.3,.3}
\definecolor{darkgray}{rgb}{.2,.2,.2}
\definecolor{shadecolor}{gray}{0.925}

%%%%%%%%%%%%%%%%%%%%%%%%%%%%%%%%%%%%%%%%%%%
% Define blocks for simple network drawings
%%%%%%%%%%%%%%%%%%%%%%%%%%%%%%%%%%%%%%%%%%%

% Define blocks for newtorks drawings
\usepackage{amsmath} % Add math­e­mat­i­cal fea­tures
\usepackage{schemabloc} % Add block diagram library (french one)

%% Define infinite bus
\tikzset{infinite bus/.pic={
  code={
  \draw (0,0) circle (2) node[inner sep=0, outer sep=0] {{$\infty$}};
  \draw (2,0) --++ (2,0);
  }
  }
}

%% Define transformer
\tikzset{transfo/.pic={
  code={
  \draw (0,0) circle (2);
  \draw (2,0) circle (2);
  \draw (4,0) --++ (4,0);
  \draw (-2,0) --++ (-4,0);
  }
  }
}

%% Define generator
\tikzset{generator/.pic={
  code={
    \draw (0,0) circle (2);
    \draw (0,0) arc (0:180:0.5);
    \draw (0,0) arc (180:360:0.5);
    \draw (-2,0) --++ (-2,0);
  }
  }
}

%% Define generator controls
\tikzset{VR/.pic={
  code={
  \draw (0,0) circle (2) node[inner sep=0, outer sep=0] {{VR}};
  }
  }
}

%% Define SVarC
\tikzset{SVarC/.pic={
  code={
  \draw (0,0) circle (4) node[inner sep=0, outer sep=0] {{SVarC}};
  }
  }
}


\begin{document}

\chapter*{Test - ``Model Generator PTanPhi"}
This is the documentation for the test ``Model Generator P TanPhi" in the Dynawo project non regression tests.

% Generic description of the non regression test
% Data origin, network size, simulation types and numbers, etc.
\section*{Test description}

This test is a basic unit test. It is based on a model (DYD
file) which instantiates a Generator PTanPhi model, an infinite bus, a line and a
load.
There are two different embedded tests  :
\begin{itemize}
\item In the first test, the generator PTanPhi is deactivated to make sure the
disconnection is properly working;
\item in the second test, the frequency oscillates and we expect the output
active power to compensate this without crossing the power limits.
The reactive power will follow the active power behavior until it reaches its limits.
\item in the third test, the active power should be doubled during 1 s until the limit is reached at 2 pu.
N.B : when using deltaPGenPu, we increase the power in percentage, when using deltaPRefPu we add the value to the active power existing.

\end{itemize}

% Description of the simulation
% Events, solver, duration, outputs, etc.
\subsection*{Simulation description}

\subsubsection*{First test}
The first simulation has the following characteristics:
\begin{itemize}
\item The switch off signal of the generator PTanPhi is enabled at $t=10s$;
\item The frequency and load of the network are constants;
\item The simulation lasts for 20s.
\end{itemize}

\subsubsection*{Second test}
The second simulation has the following characteristics:
\begin{itemize}
\item The load of the network is constant;
\item The frequency of the network oscillates around a reference value;
\item The simulation lasts for 10s.
\end{itemize}

% Expected results
% Events, logs, plots, etc.
\subsection*{Expected results}
\subsubsection*{First test}

In this test, we expect the active power to fall at 0 after the deactivation of
the generator.

The following plot illustrates this:
\begin{figure}[H]
  \caption{Evolution of active and reactive power as a function of time}
  \begin{tikzpicture}
    \begin{axis}[]
        \addplot[color=green!50]
        table[x=time,y=Generator_generator_running]
        {../reference/outputsRunning/curves/curves.csv};
        \addplot[color=blue!50]
        table[x=time,y=Generator_generator_PGenPu]
        {../reference/outputsRunning/curves/curves.csv};
        \addplot[color=red!50]
        table[x=time,y=Generator_generator_QGenPu]
        {../reference/outputsRunning/curves/curves.csv};
        \legend{$running$, $P$, $Q$}
    \end{axis}
  \end{tikzpicture}
\end{figure}

This corresponds to the following timeline:
\begin{verbatim}
  10.000000 | Generator | GENERATOR : disconnecting
\end{verbatim}

\subsubsection*{Second test}

In this test, we expect the active power to compensate the oscillations of
the frequency without crossing the minimum and maximum limits.

The following plots illustrate this:
\begin{figure}[H]
  \caption{Evolution of active power as a function of time}
  \begin{tikzpicture}
    \begin{axis}[]
        \addplot[color=green!50]
        table[x=time,y=Generator_generator_PMinPu]
        {../reference/outputsOmega/curves/curves.csv};
        \addplot[color=blue!50]
        table[x=time,y=Generator_generator_PMaxPu]
        {../reference/outputsOmega/curves/curves.csv};
        \addplot[color=red!50]
        table[x=time,y=Generator_generator_PGenPu]
        {../reference/outputsOmega/curves/curves.csv};
        \legend{$P_{Min}$, $P_{Max}$, $P$}
    \end{axis}
  \end{tikzpicture}
\end{figure}
\begin{figure}[H]
  \caption{Evolution of reactive power as a function of time}
  \begin{tikzpicture}
    \begin{axis}[]
        \addplot[color=green!50]
        table[x=time,y=Generator_generator_QMinPu]
        {../reference/outputsOmega/curves/curves.csv};
        \addplot[color=blue!50]
        table[x=time,y=Generator_generator_QMaxPu]
        {../reference/outputsOmega/curves/curves.csv};
        \addplot[color=red!50]
        table[x=time,y=Generator_generator_QGenPu]
        {../reference/outputsOmega/curves/curves.csv};
        \legend{$Q_{Min}$, $Q_{Max}$, $Q$}
    \end{axis}
  \end{tikzpicture}
\end{figure}
\begin{figure}[H]
  \caption{Evolution of frequency as a function of time}
  \begin{tikzpicture}
    \begin{axis}[]
        \addplot[color=blue!50]
        table[x=time,y=Generator_generator_omegaRefPu]
        {../reference/outputsOmega/curves/curves.csv};
        \legend{$omega_{Ref}$}
    \end{axis}
  \end{tikzpicture}
\end{figure}

This corresponds to the following timeline:
\begin{verbatim}
0.858071 | Generator | PMIN : activation
2.303528 | Generator | PMIN : deactivation
3.302162 | Generator | PTanPhi Generator : max reactive power limit reached
3.404276 | Generator | PMAX : activation
6.040505 | Generator | PMAX : deactivation
6.142616 | Generator | PTanPhi Generator : back to reactive power following active power
7.141258 | Generator | PMIN : activation
8.586713 | Generator | PMIN : deactivation
9.585347 | Generator | PTanPhi Generator : max reactive power limit reached
9.687461 | Generator | PMAX : activation
…...
\end{verbatim}

\subsubsection*{Third test}

In this test, we expect the active power to increase gradually until it reaches its limit.
The reactive power will follow until it reaches its limit.

The following plot illustrates this:
\begin{figure}[H]
  \caption{Evolution of active and reactive power as a function of time}
  \begin{tikzpicture}
    \begin{axis}[]
        \addplot[color=blue!50]
        table[x=time,y=Generator_generator_PGenPu]
        {../reference/outputsPVar/curves/curves.csv};
        \addplot[color=red!50]
        table[x=time,y=Generator_generator_QGenPu]
        {../reference/outputsPVar/curves/curves.csv};
        \legend{$running$, $PGenPu$, $QGenPu$}
    \end{axis}
  \end{tikzpicture}
\end{figure}

This corresponds to the following timeline:
\begin{verbatim}
  2.200000 | Generator | PTanPhi Generator : max reactive power limit reached
  2.333333 | Generator | PMAX : activation
\end{verbatim}

\end{document}
